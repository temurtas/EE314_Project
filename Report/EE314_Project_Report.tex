\documentclass[paper]{IEEEtran}
\IEEEoverridecommandlockouts
% The preceding line is only needed to identify funding in the first footnote. If that is unneeded, please comment it out.
\usepackage[english]{babel}
\usepackage[utf8x]{inputenc}
\usepackage{amsmath}
\usepackage{graphicx}
\usepackage[colorinlistoftodos]{todonotes}
\usepackage{gensymb} % this could be problem
\usepackage{float}
\usepackage{fancyref}
\usepackage{subcaption}
\usepackage{amssymb}

\usepackage{pythonhighlight}

\usepackage{xspace}

\newcommand\nd{\textsuperscript{nd}\xspace}
\newcommand\rd{\textsuperscript{rd}\xspace}
\newcommand\nth{\textsuperscript{th}\xspace} %\th is taken already


\usepackage{xcolor}
\usepackage{listings}

\usepackage{fancyhdr}

\usepackage{karnaugh-map}

\definecolor{mGreen}{rgb}{0,0.6,0} % for python
\definecolor{mGray}{rgb}{0.5,0.5,0.5}
\definecolor{mPurple}{rgb}{0.58,0,0.82}


\definecolor{mygreen}{RGB}{28,172,0} % color values Red, Green, Blue for matlab
\definecolor{mylilas}{RGB}{170,55,241}

\lstdefinestyle{CStyle}{
    commentstyle=\color{mGreen},
    keywordstyle=\color{magenta},
    numberstyle=\tiny\color{mGray},
    stringstyle=\color{mPurple},
    basicstyle=\footnotesize,
    breakatwhitespace=false,         
    breaklines=true,                 
    captionpos=b,                    
    keepspaces=true,                 
    numbers=left,                    
    numbersep=5pt,                  
    showspaces=false,                
    showstringspaces=false,
    showtabs=false,                  
    tabsize=2,
    language=C
}


\lstset{language=Matlab,%
    %basicstyle=\color{red},
    breaklines=true,%
    morekeywords={matlab2tikz},
    keywordstyle=\color{blue},%
    morekeywords=[2]{1}, keywordstyle=[2]{\color{black}},
    identifierstyle=\color{black},%
    stringstyle=\color{mylilas},
    commentstyle=\color{mygreen},%
    showstringspaces=false,%without this there will be a symbol in the places where there is a space
    numbers=left,%
    numberstyle={\tiny \color{black}},% size of the numbers
    numbersep=9pt, % this defines how far the numbers are from the text
    emph=[1]{for,end,break},emphstyle=[1]\color{red}, %some words to emphasise
    %emph=[2]{word1,word2}, emphstyle=[2]{style},    
}



\makeatletter
\renewcommand\paragraph{\@startsection{paragraph}{4}{\z@}%
            {-2.5ex\@plus -1ex \@minus -.25ex}%
            {1.25ex \@plus .25ex}%
            {\normalfont\normalsize\bfseries}}
\makeatother
\setcounter{secnumdepth}{4} % how many sectioning levels to assign numbers to
\setcounter{tocdepth}{4}    % how many sectioning levels to show in ToC



\begin{document}

\title{EE314 Digital Electronics Laboratory\\
2017-2018 Spring Term Project Proposal Report
}


\author{

\IEEEauthorblockN{1\textsuperscript{st} Halil TEMURTAS}
\IEEEauthorblockA{\textit{2094522} }
\textit{halil.temurtas@metu.edu.tr}

\and

\IEEEauthorblockN{2\textsuperscript{nd} Erdem TUNA}
\IEEEauthorblockA{\textit{2167419} }
\textit{erdem.tuna@metu.edu.tr}


}

\maketitle

\begin{abstract}

The project 

\end{abstract}

\begin{IEEEkeywords}
The, laboratory , project
\end{IEEEkeywords}

\section{Introduction}
\- \indent
	In this project, our aim is to design a oscilloscope.


\section{Project}
\- \indent


\begin{figure}[h!]
	\setlength{\unitlength}{\textwidth}
	\center 
	%\includegraphics[width=0.47\textwidth]{block1.png}
	\caption{\label{fig:block_diagram}The Block Diagram for the project}
\end{figure}

	 \textit{Figure~\ref{fig:block_diagram}}    



%\begin{figure}[h!]
%	\centering
%	\begin{subfigure}{.48\textwidth}
%		\centering
%		\includegraphics[width=1\linewidth]{scope_28_SimResult}
		%\caption{Theoretical Output of VCO.}
		%\label{fig:VCOFor1kHzTheoretical}
%	\end{subfigure}%
%	\newline
%	\begin{subfigure}{.48\textwidth}
%		\centering
%		\includegraphics[width=1\linewidth]{scope_28}
		%\caption{Practical Output of VCO.}
		%\label{fig:VCOFor1kHzPractical}
%	\end{subfigure}
	%\caption{VCO Outputs for 1kHz.}
%	\label{fig:VCOFor1kHz}
%\end{figure}

	
	
\begin{figure}[h!]
\setlength{\unitlength}{\textwidth}
\center 
%\includegraphics[width=0.25\unitlength]{triangular.png}
%\caption{\label{fig:trimix}One Period of Triangular Wave }
\end{figure}		
	
\section{Conclusion}
\- \indent
	Conclusion
	
	

\begin{thebibliography}{00}
\bibitem{b1} J.-J. Lin, Y.-P. Li, W.-C. Hsu, and T.-S. Lee, “Design of an FMCW radar baseband signal processing system for automotive application,” SpringerPlus, vol. 5, no. 1, 2016.	
%\bibitem{b2} “LF353 Datasheet.” [Online]. Available: http://www.ti.com/lit/ds/symlink/lf353-n.pdf. [Accessed: 20-Jan-2018].
%\bibitem{b3} “2N7000 Datasheet.” [Online]. Available: https://www.onsemi.com/pub/Collateral/2N7000-D.PDF. [Accessed: 20-Jan-2018].
%\bibitem{b4} “LF353 Datasheet.” [Online]. Available: http://www.falstad.com/circuit/e-vco.html.

%\bibitem{b6} Y. Yorozu, M. Hirano, K. Oka, and Y. Tagawa, ``Electron spectroscopy studies on magneto-optical media and plastic substrate interface,'' IEEE Transl. J. Magn. Japan, vol. 2, pp. 740--741, August 1987 [Digests 9th Annual Conf. Magnetics Japan, p. 301, 1982].
%\bibitem{b7} M. Young, The Technical Writer's Handbook. Mill Valley, CA: University Science, 1989.
\end{thebibliography}




\end{document}